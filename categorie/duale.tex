% !TEX program = lualatex
% !TEX root = ../catbook.tex
% !TEX spellcheck = it_IT

\section{Il principio di dualità}

{\color{red} [Perché non introdurre la costruzione della categoria duale qui, piuttosto che anticiparla? Così poi per parlare di dualità.]}

Partiamo con un esempio per dare una prima idea di un principio che incarna un modo di fare.

\begin{esempio}[Monomorfismi ed epimorfismi sono duali]
Come esperimento, consideriamo un monomorfismo \(f : x \to y\) di una generica categoria \(\catC\). Operiamo una traduzione nel senso dello schema che segue:
%
\begin{center}
%\begin{table}\centering
\begin{tabular}{ccc}
\toprule
\(f\) monomorfismo di \(\catC\) & & \\
\midrule
per ogni & & per ogni \\
oggetto \(a\) di \(\catC\) & \(\leftrightsquigarrow\) & oggetto \(a\) di \(\op\catC\) \\
e per ogni & & e per ogni \\
freccia \(g : a \to y\) di \(\catC\) & \(\leftrightsquigarrow\) & freccia \(g : y \to a\) di \(\op\catC\) \\
esiste al più una &  & esiste al più una \\
freccia \(h : a \to x\) di \(\catC\) & \(\leftrightsquigarrow\) & freccia \(h : x \to a\) di \(\op\catC\) \\
tale che & & tale che \\
\(g = fh\) & \(\leftrightsquigarrow\) & \(g = hf\) \\
%\bottomrule
\end{tabular}
%\caption{ciao}
%\end{table}
\end{center}
%
L'ultima colonna ci dice proprio che \(f\), vista come freccia di \(\op\catC\) questa volta, è un epimorfismo di \(\op\catC\).
\end{esempio}

\begin{esercizio}
Ragionando esattamente come abbiamo appena fatto, si può costatare facilmente come un epimorfismo in una categoria è un monomorfismo nella categoria duale.
\end{esercizio}

Il {\em principio di dualità} sancisce proprio questo: una proposizione in linguaggio categoriale vale per due, dove
%
\begin{itemize}
\item con \q{proposizione in linguaggio categoriale} intendiamo una qualsiasi proposizione sensata, costruita secondo le regole della Logica di base (connettivi e quantificatori) e che parli di oggetti, frecce e composizioni.
\item \q{vale per due} vuol dire che, siccome una categoria e la sua duale coesistono intimamente, una proposizione in linguaggio categoriale si può riformulare in termini duali; nella pratica si tratta di mantenre gli stessi oggetti, di invertire i ruoli di dominio e codominio (\q{cambia il verso delle frecce}) e di scambiare di posto le frecce nelle composizioni.
\end{itemize}

Questo significa che, ad esempio, possiamo limitarci a dare la definizione di monomorfismo (epimorfismo), per poi dire che
\begin{quotation}
una freccia \(f\) di una categoria è epimorfismo (monomorfismo) qualora \(f\) è un monomorfismo (epimorfismo) della categoria duale.
\end{quotation}

Questo principio riduce il nostro lavoro a metà oppure, equivalentemente, raddoppia i nostri risultati: un teorema in linguaggio categoriale rimane tale se lo si riscrive in termini duali e, anche qui, una dimostrazione fatta vale per due nel senso che ti aspetteresti.
