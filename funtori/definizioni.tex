% !TEX program = lualatex
% !TEX root = ../catbook.tex
% !TEX spellcheck = it_IT

\section{Definizione}

Fino ad ora abbiamo visto le categorie camminandoci dentro, mentre il passo successivo è quello di camminare fra le categorie. Per fare questo abbiamo bisogno anzitutto di una qualche nozione di frecce tra categorie: i funtori sono destinati a ricoprire questi ruoli. Il concetto di funtore deve essere pensato adeguatamente, vale a dire deve preservare della struttura. Abbiamo bisogno di qualcosa che accorpi oggetti, frecce e composizioni.

\begin{definizione}[Funtori]\label{definizione:DefFunc}
Siano \(\catC\) e \(\catD\) due categorie. Un {\em funtore}\index{funtore} \(F\) da \(\catC\) a \(\catD\) si indica con \(F : \catC \to \catD\) e consta delle seguenti funzioni indicate tutte con lo stesso nome \(F\):
\begin{itemize}
\item una funzione su oggetti \(F : \obj\catC \to \obj\catD\)
\item funzioni su frecce, ovvero per ogni \(x,y \in \obj\catC\) di una funzione
\[F : \catC(x, y) \to \catD(F(x), F(y)),\, f \to F(f)\] 
\end{itemize}
con i seguenti assiomi, che per noi sono gli {\em assiomi funtoriali} o {\em di funtorialità}\index{assiomi funtoriali}:
\begin{enumerate}
\item \(F(\id_x) = \id_{F(x)}\) per ogni oggetto \(x\) di \(\catC\)
\item \(F(gf) = F(g)F(f)\) per ogni coppia di frecce \(f\) e \(g\) di \(\catC\) componibili.
\end{enumerate}
\end{definizione}

I funtori della definizione appena data si dicono {\em covarianti}, anche se questo aggettivo è ormai desueto: quando si dice funtore, si sottointende questo aggettivo.

\begin{definizione}[Funtori controvarianti]\label{definizione:ContrFunc}
Un {\em funtore controvariante}\index{funtore controvariante} da una categoria \(\catC\) ad una categoria \(\catD\) è un funtore \(\op\catC \to \catD\).
\end{definizione}

\begin{nota}
La definizione di funtore controvariante è molto ermetica, vediamo quindi cosa la rende particolare da meritarsi un nuovo aggettivo. Un qualche funtore \(F : \op\catC \to \catD\), da definizione, fa esattamente questo: manda ogni oggetto di \(\op\catC\) in uno e un solo oggetto di \(\catD\); manda una freccia \(f : a \to b\) di \(\op\catC\) in nella \(F(f) : F(a) \to F(b)\) in \(\catD\). Ma gli oggetti di \(\op\catC\) sono esattamente quelli di \(\catC\) e una freccia \(f : a \to b\) di \(\op\catC\) in \(\catC\) è una freccia \(b \to a\). Quindi in pratica il funtore \(F : \op\catC \to \catD\) manda oggetti di \(\catC\) in oggetti di \(\catD\), e frecce \(b \to a\) di \(\catC\) in frecce \(F(a) \to F(b)\) di \(\catD\).\newline
Rimane ora da vedere osa succede agli assiomi di funtorialità. \(\catC\) e \(\op\catC\) condividono le stesse identità, quindi con le identità non fa nulla di diverso dalla definizione~\ref{definizione:DefFunc}. Prendiamo ora due frecce
\[a \functo{f} b \functo{g} c\]
di \(\op\catC\): da definizione di funtore, quindi abbiamo \(F(gf) = F(g) F(f)\). Ma le frecce prese in esame diventano
\[c \functo{g} b \functo{f} a\]
in \(\catC\), con la composizione \(gf\) di \(\op\catC\) è \(fg\) in \(\catC\). In definitiva succede quanto segue:
\[F(\underbrace{fg}_{\substack{\text{\(f\) e \(g\) come frecce di \(\catC\) e} \\ \text{composte come tali in \(\catC\)}}}) =
F(\underbrace{gf}_{\substack{\text{\(f\) e \(g\) come frecce di \(\op\catC\) e} \\ \text{composte come tali in \(\op\catC\)}}}) =
F(g)F(f).\]
\end{nota}

\begin{esercizio}
Cos'è un funtore \(\catC \to \op\catD\)?
\end{esercizio}

\begin{costruzione}[Categoria delle categorie?]
C'è un modo sensato di comporre due funtori: se \(F : \catA \to \catB\) e \(G : \catB \to \catC\) sono funtori, la composizione \(GF : \catA \to \catC\) è così fatta:
%
\begin{itemize}
\item manda un oggetto \(a\) di \(\catA\) nell'oggetto \(G(F(a))\) di \(\catC\)
\item manda una freccia \(f : x \to y\) di \(\catA\) nella freccia
\[G(F(f)) : G(F(x)) \to G(F(y))\]
di \(\catC\).
\end{itemize}
%
È immediato verificare che questa composizione di funtori è associativa, e il funtore identità \(\id_\catA : \catA \to \catA\) su una categoria \(\catA\) fa quello che ti aspetteresti. Ci sono insomma i presupposti per formare una \q{categoria delle categorie}, solo che si rischia di incappare in cavilli fondazionali! Immagina infatti che tra gli oggetti di questa categoria ce ne sia una (e perché dovrebbe non esserci?) i cui oggetti formano una classe propria: se si costruisce la Teoria delle Categorie sopra la Teoria degli Insiemi secondo NBG, questo non è possibile, vedi~\cite{berarducci:insiemi}. Ci sono tuttavia altre Teorie degli Insiemi che permettono la \q{categoria di tutte le categorie}, come fatto in~\cite{adamek-herrlich-strecker:cats}.
\end{costruzione}

Vogliamo avere qualcosa come nell'esempio~\ref{esempio:ElemAsFunc}.

\begin{definizione}[Elemento di una categoria]\label{definizione:CatElem}
I funtori \(\uno \to \catC\) si chiamano {\em elementi}\index{elemento} di \(\catC\). Adotteremo la seguente convenzione: se \(x\) è un oggetto di \(\catC\), allora \(x\) indica anche l'elemento di \(\catC\) che manda l'unico oggetto di \(\uno\) in \(x\) e l'unica identità di \(\uno\) in \(\id_x\). 
\end{definizione}
