% !TEX program = lualatex
% !TEX root = ../catbook.tex
% !TEX spellcheck = it_IT

\chapter{Sandbox}

Questa sezione come da titolo è un mondo a parte: più che scollegata dal resto dell'opera, raccoglie idee che mi vengono di volta in volta e che mi annoto per non farmele sfuggire. È per questo che non ci sono definizioni, teoremi, esempi, od altro ma {\em flussi di coscienza}.

Volendo si può pensare così: se qualcosa è in questa sezione, vuol dire almeno una delle seguenti:
\begin{itemize}
\item ha bisogno di affinamenti, ovvero non è ancora matura a sufficienza, se non a volta il discorso è interrotto bruscamente a metà
\item non ha trovato una collocazione all'interno della trattazione che può essere inesistente, incompleta o in attesa di una revisione.
\end{itemize}

È inutile aspettarsi da questa sezione una certa coerenza o consequenzialità.

Stando così, penso che questa sezione verrà mantenuta finché non si sarà svuotata del tutto.


\section{Funtori}




\section{Limiti e colimiti}

\begin{flusso}[Reticoli]
Un {\em reticolo} \((R, \leqslant)\) è un poset avente per ogni \(x, y \in R\) il prodotto \(x \land y\) e il coprodotto \(x \lor y\). Un reticolo è {\em limitato} qualora abbia un iniziale \(0\) e un terminale \(1\).
\end{flusso}

\begin{flusso}[Algebra di Heyting]
Un'{\em algebra di Heyting} \(\bfH\) è un reticolo con un'operazione \(\rightfree\) tale che per ogni \(a, b, c \in \bfH\)
\[a \land b \leqslant c \text{ se e solo se } a \leqslant (b \rightfree c)\,.\]
\end{flusso}