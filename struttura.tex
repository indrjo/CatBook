% !TEX program = lualatex
% !TEX root = ./catbook.tex

\documentclass[ structure      = book
              , maketitlestyle = standard
              , chapstyle      = parcenter
              , tocchapfont    = smallcaps
              , secstyle       = left
              , secfont        = roman
              , liststyle      = aligned
              , quotesize      = normalsize
              , headerstyle    = center
              ]{suftesi}

% eliminare alcuni noiosi warnings...
%\usepackage{silence}
%\WarningFilter[xcol]{xcolor}{Incompatible color definition}
%\ActivateWarningFilters[xcol]

% dimensioni dei glifi flessibili
\usepackage{anyfontsize}

% gestione dei font
\usepackage[no-math]{fontspec}
\setmainfont{Linux Libertine O}

% gestione lingue
\usepackage{polyglossia}
\setmainlanguage{italian}

% virgolette
\usepackage[italian = quotes]{csquotes}
\newcommand\q\enquote

% riferimenti incrociati
\usepackage{hyperref}
\hypersetup{ breaklinks
           , hidelinks
           , linktoc = page
           }
\newcommand\mail[1]{\href{mailto:#1}{{\ttfamily #1}}}

% bibliografia
\usepackage[style = alphabetic-verb]{biblatex}
\addbibresource{mybib.bib}

% indice analitico
\usepackage{imakeidx}
\makeindex[columns=2]

% tabelle
\usepackage{booktabs}

% liste
%\setlist[itemize]{ label = \textbullet }

% didascalie immagini
%\captionsetup{ format = hang
%             , font   = small
%}

% testo su più colonne
\usepackage{multicol}

%%% MATEMATICA

%% Font
\usepackage{ libertinust1math
           , MnSymbol
           , mathtools
           , extarrows
           , amsthm
           }
\usepackage[bb=ams,cal=cm,calscaled=.94]{mathalfa}

%% Definizioni, teoremi, esempi, esercizi, osservazioni...
\newcounter{cat}                           % nuovo contatore
\counterwithin{cat}{chapter}
\theoremstyle{definition}                  % stile usato per amsthm
\newtheorem{definizione}[cat]{Definizione} % definizioni
\newtheorem{costruzione}[cat]{Costruzione} % costruzioni
\newtheorem{teorema}[cat]{Teorema} 	       % teoremi
\newtheorem{esempio}[cat]{Esempio}         % esempi
\newtheorem{nota}[cat]{Nota}               % note
\newtheorem{notazione}[cat]{Notazione}     % notazioni
\newtheorem{esercizio}[cat]{Esercizio}     % esercizi
\newtheorem{evasione}[cat]{Evasione}       % evasioni

\newcommand\nextcat{\stepcounter{cat}\arabic{chapter}.\arabic{cat}}

\newcommand\perlettore{{\scshape Esercizio~\nextcat}}

\renewcommand\theequation{\thechapter.\arabic{equation}}

%% Diagrammi commutativi
\usepackage{tikz}
\usetikzlibrary{ calc
               , babel
               , cd
               }
\tikzcdset{ row sep/normal=.7cm
          , column sep/normal=.7cm
          , arrow style=tikz
          , diagrams={>={To[length=3.5pt,width=4pt]}}
          , shorten=-2pt                             
          , natural/.style={shorten=2pt, Rightarrow}
          , u func/.style={bend left=35, r}
          , l func/.style={bend right=35, r, swap}
          }

\usepackage{bbding}
