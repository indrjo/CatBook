% !TEX program = lualatex
% !TEX root = ../catbook.tex
% !TEX spellcheck = it_IT

\section{Limiti e colimiti}

\begin{definizione}[Limiti e colimiti]
Siano \(\catC\) ed \(\bfI\) due categorie.
%
\begin{itemize}
\item Un {\em diagramma} di forma \(\bfI\) è un funtore \(D : \bfI \to \catC\). {\color{red} [Un\dots{} {\em codiagramma} o {\em diagramma duale}?]}
\item Sia \(v \in \obj\catC\) e il funtore \(k_v : \bfI \to \catC\) che manda qualsiasi oggetto di \(\bfI\) in \(v\) e qualsiasi freccia di \(\bfI\) in \(\id_v\). Un {\em cono} sopra \(D\) di vertice \(v\) è una qualsiasi trasformazione naturale
\[\begin{tikzcd}
\bfI \ar["{k_v}"{name=kv}, u func] \ar["D"{name=D}, l func] & \catC
\ar[natural, from=kv, to=D]
\end{tikzcd}.\]
Dualmente, un {\em cocono} sotto \(D\) di vertice \(v\) è una qualsiasi trasformazione naturale
\[\begin{tikzcd}
\bfI \ar["D"{name=D}, u func] \ar["{k_v}"{name=kv}, l func] & \catC
\ar[natural, from=D, to=kv]
\end{tikzcd}.\]
{\color{red} [Dare la definizione di cocono attraverso diagrammi duali?]}
\item Un {\em limite} di \(D\) è un cono \(\set{L \functo{\lambda_i} D(i)}_{i \in \obj\bfI}\) su \(D\) tale che per ogni cono \(\set{A \functo{\alpha_i} D(i)}_{i \in \obj\bfI}\) su \(D\) si ha che esiste una e una sola freccia \(h : A \to L\) di \(\catC\) per la quale commuta
\[\begin{tikzcd}[row sep=tiny]
A \ar["{\alpha_i}", dr] \ar["h", swap, dd] &      \\
                                           & D(i) \\
L \ar["{\lambda_i}", swap, ur]
\end{tikzcd}\]
per ogni \(i \in \obj\bfI\). {\color{red} [Da dare ancora la nozione di colimite\dots{}]}
\end{itemize}
%
\end{definizione}
