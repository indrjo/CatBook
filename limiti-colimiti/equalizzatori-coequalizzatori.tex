% !TEX program = lualatex
% !TEX root = ../catbook.tex
% !TEX spellcheck = it_IT

\section{Equalizzatori e coequalizzatori}

\begin{definizione}[Equalizzatori e coequalizzatori]
{\color{red} [Generalizzare con una generica collezione di frecce parallele?]}
Sia \(\catC\) una categoria e \(a, b \in \obj\catC\). Un {\em equalizzatore} in \(\catC\) di due frecce
\begin{equation}\begin{tikzcd}
a \ar["f", r, shift left] \ar["g", swap, r, shift right] & b
\end{tikzcd}\label{eqn:ParArr}\end{equation}
di \(\catC\) è un \(e \in \obj\catC\) con una freccia \(i : e \to a\) tale che \(fi = gi\) con la seguente proprietà: per ogni \(d \in \obj\catC\) con una freccia \(j : d \to a\) di \(\catC\) tale che \(fj = gj\) esiste una e una sola freccia \(k : d \to e\) di \(\catC\) per cui commuta
\[\begin{tikzcd}[row sep=tiny]
d \ar["j", dr] \ar["k", swap, dd] &   \\
                                  & a \\
e \ar["i", swap, ur]              &          
\end{tikzcd}\]
Abbiamo il concetto duale: un {\em coequalizzatore} di~\eqref{eqn:ParArr} all'interno di \(\catC\) è esattamente un equalizzatore delle stesse frecce viste come frecce di \(\op\catC\).
\end{definizione}

\begin{esempio}
Siamo in \(\Set\): presi due insiemi \(A\) e \(B\) e due funzioni
\[\begin{tikzcd}
A \ar["f", r, shift left] \ar["g", swap, r, shift right] & B
\end{tikzcd}\]
abbiamo l'{\em equalizzatore} delle due funzioni
\[\eq(f, g) \coloneq \set{x \in A \mid f(x) = g(x)}.\]
Il nome suggerisce che è un equalizzatore nel senso che vogliamo: \(\eq(f, g)\) con l'inclusione \(i : \eq(f, g) \hookrightarrow A\) è un equalizzatore di \(f\) e \(g\). Infatti presa un qualsiasi insieme \(C\) con una funzione \(j : C \to A\) tale che \(fj = gj\), la funzione
\[h : C \to \eq(f, g)\,, \ h(x) = j(x)\]
fa al caso nostro, ovvero fa commutare
\[\begin{tikzcd}[row sep=small]
C \ar["j", dr] \ar["h", swap, dd] \\
 & A \\
\eq(f, g) \ar["i", swap, ur]
\end{tikzcd}.\]
(Notare che la definizione di \(h\) è proprio possibile in quanto \(fj = gj\).) È immediato vedere che \(h\) è l'unica per cui le cose funzionano.
\end{esempio}

\begin{esercizio}[Pullbacks ed equalizzatori]
Sia \(\catC\) una categoria. Se
\[\begin{tikzcd}
e \ar["i", r] \ar["i", swap, d] & a \ar["f", d] \\
a \ar["g", swap, r]             & b
\end{tikzcd}\]
è un suo quadrato di pullback, allora \(e\) con \(i\) è un equalizzatore di \(f\) e \(g\)? È vero il viceversa? Esiste un'ovvia dualizzazione di questo esercizio.
\end{esercizio}