% !TEX program = lualatex
% !TEX root = ../catbook.tex
% !TEX spellcheck = it_IT

\section{Pullbacks e pushouts}

\begin{definizione}[Pullbacks e pushouts]
{\color{red} [Pullbacks (pushouts) di collezioni di frecce aventi codominio (dominio) comune?]}
Sia \(\catC\) una categoria: un {\em pullback}\index{pullback} in \(\catC\) di una coppia di frecce di \(\catC\) con codominio comune
\[\begin{tikzcd}
                    & a \ar["s", d] \\
b \ar["t", r, swap] & c
\end{tikzcd}\]
è un qualsiasi oggetto \(p\) di \(\catC\) con una coppia di frecce \(f : p \to a\) e \(g : p \to b\) di \(\catC\) tali da far commutare
\begin{equation}\begin{tikzcd}
p \ar["f", r] \ar["g", d, swap] & a \ar["s", d] \\
b \ar["t", r, swap]             & c
\end{tikzcd}\label{eqn:PullbackSquare}\end{equation}
e con la seguente proprietà: per ogni oggetto \(q\) di \(\catC\) con frecce \(h : q \to a\) e \(k : q \to b\) di \(\catC\) per le quali commuta
\[\begin{tikzcd}
q \ar["h", r] \ar["k", d, swap] & a \ar["s", d] \\
b \ar["t", r, swap]             & c
\end{tikzcd}\]
esiste una e una sola freccia \(l : q \to p\) di \(\catC\) tale che \(fl=h\) e \(gl=k\).
%
Il quadrato\footnote{Nulla di strettamente geometrico, quattro oggetti e quattro frecce}~\eqref{eqn:PullbackSquare} è detto \q{di pullback}\index{quadrato di pullback}. Dualmente, un {\em pushout}\index{pushout} in \(\catC\) di due frecce
\[\begin{tikzcd}
c \ar["s", r] \ar["t", d, swap] & a \\
b                               &
\end{tikzcd}\]
di \(\catC\) è uno qualsiasi dei pullback in \(\op\catC\) della stessa coppia di frecce, questa volta viste come frecce di \(\op\catC\). Esiste anche un'ovvia nozione di \q{quadrato di pushout}\index{quadrato di pushout}.
\end{definizione}

%\begin{figure}
%% !TEX root = ../catbook.tex
% !TEX program = lualatex

\begin{tikzcd}
\forall q \ar["{\forall h}", ddr, bend right=20, swap] \ar["{\forall k}", drr, bend left=20] \ar["{\exists! l}"{description}, dr] & \\
 & p \ar["f", d] \ar["g", r, swap] & b \ar["t", d] \\
 & a \ar["s", r, swap]             & c 
\end{tikzcd}
%\caption{\(p\) con le frecce \(a \functo{f} p \lfuncto{g} b\) è un pullback della coppia di frecce \(a \xrightarrow{\ s \ } c \xleftarrow{\ t \ } b\).}
%\end{figure}

\begin{esercizio}
Come abbiamo fatto nel paragrafo precedente, costruire una categoria apposita in cui un pullback (pushout) risulta un oggetto terminale (iniziale). In altre parole costruirai la {\em categoria dei coni} su
\[\begin{tikzcd}
         & a \ar[d] \\
b \ar[r] & c  
\end{tikzcd}.\]
Fatto questo, puoi dire che un pullback (pushout) di una stessa coppia di frecce è unico a meno di isomorfismi. Anche in questo caso, osserva che un isomorfismo di coni è un isomorfismo di vertici.
\end{esercizio}

\begin{esempio}[Le preimmagini]
Sia \(f : X \to Y\) una funzione di insiemi e \(C \subseteq Y\). Facciamo vedere che
\[\begin{tikzcd}
\inv f C \ar["{\bar f}", r] \ar["{i_X}", swap, d] & C \ar["{i_Y}", d] \\
X \ar["f", swap, r]                               & Y 
\end{tikzcd}\]
è un quadrato di pullback di \(\Set\), dove \(\bar f (c) \coloneq f(c)\), \(i_X (a) \coloneq a\) e \(i_Y (b) \coloneq b\).\newline
Anzitutto è facile vedere che il quadrato sopra commuta. Prendiamo un qualsiasi insieme \(A\) con due funzioni \(h : A \to X\) e \(k : A \to C\) per le quali commuta
\[\begin{tikzcd}
A \ar["k", r] \ar["h", swap, d] & C \ar["{i_Y}", d] \\
X \ar["f", swap, r]             & Y 
\end{tikzcd}:\]
la funzione \(j : A \to \inv f C\) tale che \(j(z) \coloneq h(z)\) soddisfa \(h = i_X j\) e \(k = \bar f j\) (fai i conti!). Sia ora una funzione \(j' : A \to \inv f C\) tale che \(h = i_X j'\) e \(k = \bar f j'\): dalla prima di queste due abbiamo che \(h(x) = i_X j'(x) = j'(x)\) per ogni \(x \in A\) e quindi \(j' = j\).
\end{esempio}

\begin{esercizio}[La funzione caratteristica]
Dati un insieme \(X\) e \(A \subseteq X\), la {\em funzione caratteristica} di \(A\) è
\[\chi_A : X \to 2\,, \ \chi_A(x) \coloneq \begin{cases}1 & \text{se } x \in A \\ 0 & \text{altrimenti}\end{cases}\]
nella quale \(2 \coloneq \set{0, 1}\). Abbiamo così il quadrato di \(\Set\)
\begin{equation}\begin{tikzcd}
A \ar["i", r] \ar["{\exists!}", d, swap] & X \ar["{\chi_A}", d] \\
1 \ar["{\downvdash}", r, swap]           & 2
\end{tikzcd},\label{eqn:CharFunc}\end{equation}
dove \(\downvdash\) manda l'unico elemento del singoletto \(1\) in \(1 \in 2\). Provare che~\eqref{eqn:CharFunc} è un quadrato di pullback in \(\Set\).
\end{esercizio}

\begin{teorema}[Lemma dei pullbacks, o PBL\footnote{Sigla dell'espressione inglese \q{Pullback Lemma}.}]\label{teorema:PBL}
Si consideri il seguente diagramma
\[\begin{tikzcd}
\bullet \ar["a", r] \ar["d", d, swap] & \bullet \ar["b", r] \ar["g", d] & \bullet \ar["c", d]\\
\bullet \ar["e", r, swap] & \bullet \ar["f", r, swap] & \bullet
\end{tikzcd}\]
in cui il rettangolo esterno commuta e il quadrato a destra è di pullback. Il quadrato a sinistra è di pullback se e solo se il rettangolo esterno lo è.
\end{teorema}

\begin{proof}
Assumiamo che il quadrato a sinistra sia di pullback e sia \(\ast\) un oggetto qualsiasi e \(h, k\) due frecce tali che il rettangolo in colore qui sotto
%
\[\begin{tikzcd}
\ast \ar["h", bend left=10, drrr, blue] \ar["k", bend right=10, ddr, swap, blue] & & & \\
& \bullet \ar["a", r, swap] \ar["d", d] & \bullet \ar["b", r, swap] \ar["g", d] & \bullet \ar["c", d, blue]\\
& \bullet \ar["e", r, swap, blue] & \bullet \ar["f", r, swap, blue] & \bullet
\end{tikzcd}\]
%
commuta, ovvero \(ch=f(ek)\). Da ipotesi, però \(\bullet \lfuncto{b} \bullet \functo{g} \bullet\) è un pullback di \(\bullet \functo{f} \bullet \lfuncto{c} \bullet\), e quindi esiste una e una sola freccia \(l\) tale che \(h = bl\) e \(gl = ek\):
%
\[\begin{tikzcd}
\ast \ar["h", bend left=10, drrr, blue] \ar["k", bend right=10, ddr, swap, blue] \ar["l"{description}, bend left=8, drr] & & & \\
& \bullet \ar["a", r, swap] \ar["d", d] & \bullet \ar["b", r, swap] \ar["g", d] & \bullet \ar["c", d, blue]\\
& \bullet \ar["e", r, swap, blue] & \bullet \ar["f", r, swap, blue] & \bullet
\end{tikzcd}\]
%
Abbiamo appena detto anche che quanto evidenziato qui sotto commuta
%
\[\begin{tikzcd}
\ast \ar["h", bend left=10, drrr] \ar["k", bend right=10, ddr, swap, red] \ar["l"{description}, drr, bend left=8, red] & & & \\
& \bullet \ar["a", r, swap] \ar["d", d] & \bullet \ar["b", r, swap] \ar["g", d, red] & \bullet \ar["c", d]\\
& \bullet \ar["e", r, swap, red] & \bullet \ar["f", r, swap] & \bullet
\end{tikzcd}\]
%
Di nuovo, essendo da ipotesi \(\bullet \lfuncto{a} \bullet \functo{d} \bullet\) un pullback di \(\bullet \functo{e} \bullet \lfuncto{g} \bullet\), esiste una e una sola freccia \(m\) tale che \(l = am\) e \(k = dm\):
%
\[\begin{tikzcd}
\ast \ar["h", bend left=10, drrr] \ar["k", bend right=10, ddr, swap, red] \ar["l"{description}, drr, bend left=8, red] \ar["m"{description}, dr] & & & \\
& \bullet \ar["a", r, swap] \ar["d", d] & \bullet \ar["b", r, swap] \ar["g", d, red] & \bullet \ar["c", d]\\
& \bullet \ar["e", r, swap, red] & \bullet \ar["f", r, swap] & \bullet
\end{tikzcd}\]
%
Ci sono tutte le carte in gioco per dire che \(h = bam\) e \(k = dm\), e che le nostre costruzioni ci portano necessariamente a \(m\): pertanto \(\bullet \lfuncto{d} \bullet \functo{ba} \bullet\) è un pullback di \(\bullet \functo{fe} \bullet \lfuncto{c} \bullet\).\newline
%L'implicazione nel senso opposto è lasciata al lettore come esercizio (\perlettore).
\end{proof}

\begin{esercizio}
Completare la dimostrazione del teorema~\ref{teorema:PBL} provado l'implicazione restante.
\end{esercizio}

\begin{esercizio}
Provare: \(f : a \to b\) è un monomorfismo se e solo se
\[\begin{tikzcd}
a \ar["{\id_a}", r] \ar["{\id_a}", d, swap] & a \ar["f", d] \\
a \ar["f", r, swap]                         & b
\end{tikzcd}\]
è un quadrato di pullback.
\end{esercizio}

\begin{esercizio}
Considera il quadrato di pullback
\[\begin{tikzcd}
\bullet \ar["f", r] \ar[d] & \bullet \ar[d] \\
\bullet \ar["g", r, swap]  & \bullet
\end{tikzcd}:\]
dimostra che se \(g\) è un monomorfismo, allora anche \(f\) lo è.
\end{esercizio}

\begin{teorema}
Sia \(\catC\) una categoria, \(1 \in \obj\catC\) oggetto terminale, \(a, b , p \in \obj\catC\) e due frecce  di \(\catC\)
\[\begin{tikzcd}
a & p \ar["{\pi_1}", swap, l] \ar["{\pi_2}", r] & b
\end{tikzcd}\]
Sono equivalenti:
\begin{enumerate}
\item \(p\) assieme a \(\pi_1\) e \(\pi_2\) è un prodotto di \(a\) e \(b\) in \(\catC\).
\item \(p\) con \(\pi_1\) e \(\pi_2\) è un pullback in \(\catC\) di \(a \to 1 \gets b\).
%È di pullback il quadrato di \(\catC\)
%\[\begin{tikzcd}
%p \ar["{\pi_1}", r] \ar["{\pi_2}", swap, d] & a \ar["\exists!", d] \\
%b \ar["\exists!", swap, r]              & 1
%\end{tikzcd}.\]
\end{enumerate}
\end{teorema}

\begin{proof}
Sia
\[\begin{tikzcd}
c \ar["f", r] \ar["g", swap, d] & a \ar["\exists!", d] \\
b \ar["\exists!", swap, r]              & 1
\end{tikzcd}\]
un quadrato di \(\catC\). Questo commuta perché \(1\) è oggetto terminale; se poi \(p\) con le due frecce \(\pi_1\) e \(\pi_2\) è un prodotto in \(\catC\) di \(a\) e \(b\), allora abbiamo che esiste una e una sola freccia \(h : c \to p\) tale che \(\pi_1h = f\) e \(\pi_2h = g\).\newline
Viceversa, per ogni \(c \in \obj\catC\) con due frecce
\[a \lfuncto{f} c \functo{g} b\]
di \(\catC\) si ha il quadrato commutativo
\[\begin{tikzcd}
c \ar["f", r] \ar["g", swap, d] & a \ar["\exists!", d] \\
b \ar["\exists!", swap, r]              & 1
\end{tikzcd}\]
di \(\catC\): da definizione, esiste una e una sola freccia \(k : c \to p\) tale che \(\pi_1h = f\) e \(\pi_2h = g\). Trascurando le due frecce puntanti ad \(1\), abbiamo costruito all'interno di \(\catC\) un prodotto di \(a\) e \(b\).
\end{proof}

\begin{esercizio}
Dualizza il teorema precedente.
\end{esercizio}

\begin{esercizio}
Cosa sono i pullbacks e i pushouts dell'insieme ordinato \((\wp X, \subseteq)\), per \(X\) insieme?
\end{esercizio}
