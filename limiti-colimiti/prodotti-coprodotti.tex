% !TEX program = lualatex
% !TEX root = ../catbook.tex
% !TEX spellcheck = it_IT

\section{Prodotti e coprodotti}

\begin{definizione}[Prodotti e coprodotti]
Sia \(\catC\) una categoria e \(\set{x_i}_{i \in I}\) una collezione di suoi oggetti. Un oggetto \(\bar x\) di \(\catC\) con una collezione di frecce \(\set{\prj_i : \bar x \to x_i}_{i \in I}\) di \(\catC\) è un {\em prodotto}\index{prodotto} in \(\catC\) di \(\set{x_i}_{i \in I}\) qualora per ogni oggetto \(a\) di \(\catC\) con una collezione \(\set{f_i : a \to x_i}_{i \in I}\) di frecce di \(\catC\) esiste una e una sola freccia \(h : a \to \bar x\) di \(\catC\) per cui
\[\begin{tikzcd}[row sep=tiny]
a \ar["{f_i}", dr] \ar["h",dd, swap] &    \\
                                    & x_i \\
\bar x \ar["{\prj_i}", ur, swap]       &
\end{tikzcd}\]
commuta per ogni \(i \in I\). Un oggetto \(\underline x\) di \(\catC\) con una collezione \(\set{\inj_i : x_i \to \underline x}_{i \in I}\) di frecce di \(\catC\) è un {\em coprodotto}\index{coprodotto} in \(\catC\) di \(\set{x_i}_{i \in I}\) quando è un prodotto in \(\op\catC\) di \(\set{x_i}_{i \in I}\).
\end{definizione}

\begin{esercizio}
Quindi che cos'è un coprodotto, se dovessi spendere molte più parole per definirlo?
\end{esercizio}

Qui uno potrebbe aspettarsi un qualcosa del tipo \q{i prodotti (coprodotti) sono isomorfi tra di loro}, ma ci sono dei problemi non da poco: per parlare di isomorfismo ci dovrebbe essere, a monte di tutto, una nozione di freccia da un prodotto all'altro; ed un prodotto di una collezione \(\set{x_i}_{i \in I}\) è fatto di un oggetto \(\bar x\) e di una collezione di frecce uscenti \(f_i : \bar x \to x_i\), e quindi è il solo oggetto \(\bar x\). Il problema si risolve costruendo una categoria opportuna in cui accade quello che vogliamo.

\begin{costruzione}[Coni sopra una collezione]
Sia \(\catC\) una categoria. La {\em categoria dei coni} sopra una collezione \(E \subseteq \obj\catC\) è fatta così:
%
\begin{itemize}
\item gli oggetti sono gli oggetti \(a\) di \(\catC\) con una collezione \(\set{f_x : a \to x}_{x \in E}\) di frecce di \(\catC\) uscenti da \(a\): questa cosa si chiama {\em cono} di vertice \(a\) su \(E\) e la si indica con \(\set{f_x : a \to x}_{x \in E}\);
%
\item le frecce da un cono \(\set{f_x : a \to x}_{x \in E}\) ad un cono \(\set{g_x : b \to x}_{x \in E}\) sono tutte e sole le frecce \(h : a \to b\) di \(\catC\) per cui per ogni \(x \in E\) commuta
\[\begin{tikzcd}[row sep=tiny]
a \ar["{f_x}", dr] \ar["h",dd, swap] &   \\
                                     & x \\
b \ar["{g_x}", ur, swap]             &   \\
\end{tikzcd}\]
%
\item la composizione di due frecce
\[\set{f_x : a \to x}_{x \in E} \xrightarrow{\ h\ }
\set{g_x : b \to x}_{x \in E} \xrightarrow{\ k\ }
\set{h_x : c \to x}_{x \in E}\]
è esattamente la composizione \(kh : a \to c\) di \(\catC\).
\end{itemize}
%
La verifica degli assiomi categoriali è immediata.
\end{costruzione}

La definizione di prodotto viene facilmente riformulata in questo modo: un prodotto di \(\set{x_i}_{i \in I}\) è esattamente un oggetto terminale della categoria dei coni su \(\set{x_i}_{i \in I}\). Questo basta per dire a rigore che i prodotti su una stessa collezione sono tra loro isomorfi.

\begin{esercizio}[Coconi sotto una collezione]
Emula quanto fatto nella costruzione precedente: costruisci una categoria in cui un coprodotto può essere visto come oggetto iniziale.
\end{esercizio}

Osservando le precedenti costruzioni, è importante notare questo: un isomorfismo di (co)coni, induce un isomorfismo dei rispettivi vertici; mentre, il viceversa non sempre accade.

\begin{esercizio}
Abbiamo visto che i prodotti (coprodotti) sono oggetti terminali (iniziali) di una categoria ad hoc. Invero, vale anche il viceversa: i terminali (iniziali) possono essere visti come prodotti (coprodotti).
\end{esercizio}

\begin{esercizio}
Cos'è il (co)prodotto di una collezione con un solo elemento?
\end{esercizio}

\begin{esempio}
Facciamo vedere come il prodotto cartesiano di due insiemi sia prodotto in senso categoriale (e quindi in \(\Set\) i prodotti di due oggetti ci sono).\newline
Se \(A\) e \(B\) due insiemi, dall'insieme \(A \times B\) partono le funzioni {\em componenti}:
\begin{align*}
& p_A : A \times B \to A\,, \ (x, y) \to x \\
& p_B : A \times B \to B\,, \ (x, y) \to y\,.
\end{align*}
Consideriamo ora un qualsiasi insieme \(C\) con una coppia di funzioni uscenti \(f : C \to A\) e \(g : C \to B\): la funzione \(h : C \to A \times B\), \(h(x) \coloneq (f(x), g(x))\) è tale da far commutare
\[\begin{tikzcd}[row sep=tiny]
  & C \ar["f", dl, swap] \ar["g", dr] \ar["h", dd]     &   \\
A &                                                    & B \\
  & A \times B \ar["{p_A}", ul] \ar["{p_B}", ur, swap] &
\end{tikzcd}\]
Facciamo vedere che è l'unica a farlo: a tal scopo sia anche \(h' : C \to A \times B\) tale che \(f = p_A h'\) e \(g = p_B h'\). Abbiamo così che per ogni \(x \in C\)
\[h(x) = (f(x), g(x)) = (p_A h'(x),p_B h'(x)) = h'(x).\]
In definitiva, \(A \times B\) con le due funzioni \(p_A\) e \(p_B\) è un prodotto di \(\set{A, B}\).\newline
Secondo la Teoria degli Insiemi, il coprodotto di due insiemi \(A\) e \(B\) è l'insieme
\[A + B \coloneq (A \times \set{A}) \cup (B \times \set{B}).\]
Questo insieme se viene con due funzioni piuttosto naturali
\begin{align*}
& q_A : A \to A+B\,, \ q_A(x) \coloneq (x, A) \\
& q_B : B \to A+B\,, \ q_B(x) \coloneq (x, B)\,.
\end{align*}
Sia ora un qualsiasi insieme \(C\) e due funzioni \(f : A \to C\) e \(g : B \to C\): consideriamo così la funzione
\[h : A+B \to C\,, \ h(x, A) \coloneq f(x)\,, \ h(x, B) \coloneq g(x).\]
Questa funzione fa commutare
\[\begin{tikzcd}[row sep=tiny]
                                      & C                & \\
A \ar["{q_A}", dr, swap] \ar["f", ur] &                  & B \ar["{q_B}", dl] \ar["g", ul, swap] \\
                                      & A+B \ar["h", uu] & 
\end{tikzcd}\]
Come ti aspetteresti, facciamo vedere che solo \(h\) fa questo. Sia \(h' : A+B \to C\) per cui \(f = h' q_A\) e \(g = h' q_B\). È immediato infatti verificare che \(h = h'\):
\begin{align*}
& h(x, A) = f(x) = h'(x, A) \text{ per ogni } x \in A \\
& h(x, B) = g(x) = h'(x, B) \text{ per ogni } x \in B\,.
\end{align*}
Quindi, in questo caso l'insieme \(A+B\) con la coppia di funzioni \(q_A\) e \(q_B\) è un coprodotto di \(\set{A, B}\).
\end{esempio}

\begin{esercizio}
Quindi cos'è una coppia ordinata?
\end{esercizio}

\begin{esercizio}
In generale, il prodotto cartesiano di una famiglia di insiemi \(\set{X_i}_{i \in I}\) è l'insieme
\[\prod_{i \in I} X_i \coloneq \set{\left. f : I \to \bigcup_{i \in I} X_i \right\mid f(i) \in X_i \text{ per ogni } i \in I}\,,\]
i cui elementi si chiamano \(I\)-uple.\footnote{Nel caso finito, prendendo \(I = \set{1, \dots{}, n}\), una \(I\)-upla \(f\) viene scritta sotto forma di \(n\)-upla \((f(1), \dots{}, f(n))\).} A questa si associano le funzioni
\[p_k : \prod_{i \in I} X_i \to X_k\,, \ p_k(f) \coloneq f(k) \quad \text{per } k \in I.\]
L'insieme \(\prod_{i \in I} X_i\) affiancato dalle frecce \(\set{p_i}_{i \in I}\) è un prodotto di \(\set{X_i}_{i \in I}\)?\newline
Nella Teoria degli Insiemi il coprodotto di una famiglia in insiemi \(\set{X_i}_{i \in I}\) è
\[\coprod_{i \in I} X_i \coloneq \bigcup_{i \in I} X_i \times \set{i}.\]
Questo insieme se ne viene con funzioni piuttosto naturali, per \(j \in I\),
\[q_j : X_j \to \coprod_{i \in I} X_i\,, \ q_j(x) \coloneq (x, j).\]
Anche qui, vedere se \(\coprod_{i \in I} X_i\) con \(\set{q_i}_{i \in I}\) è un coprodotto di \(\set{X_i}_{i \in I}\).
\end{esercizio}

\begin{esercizio}
Sia \(\set{X_i}_{i \in I}\) una famiglia di spazi topologici.\newline
Lo spazio topologico prodotto di \(\set{X_i}_{i \in I}\) è l'insieme \(\prod_{i \in I} X_i\) con la {\em topologia di Tychonoff}, definita come la più piccola delle topologie su \(\prod_{i \in I} X_i\) che rendono continue le funzioni componenti \(p_i\), per \(i \in I\). Mostrare che lo spazio \(\prod_{i \in I} X_i\) con le funzioni continue \(\set{p_i}_{i \in I}\) è un prodotto in \(\Top\). {\color{red} [Non è meglio parlare del prodotto di due spazi topologici, piuttosto che mettere tutto sul generale?]}\newline
La {\em topologia dell'unione disgiunta}, invece, è la più grande delle topologie su \(\coprod_{i \in I} X_i\) per cui sono continue le iniezioni \(q_i\), al variare di \(i \in I\): come prima, provare che \(\coprod_{i \in I} X_i\) con la topologia dell'unione disgiunta e le iniezioni \(\set{q_i}_{i \in I}\) è un coprodotto in \(\Top\). {\color{red} [Stessa cosa.]}
\end{esercizio}

\begin{esercizio}
Sia \(V\) uno spazio vettoriale su campo \(k\) e \(V_1, \dots{}, V_n\) suoi sottospazi: abbiamo così lo {\em spazio vettoriale somma}
\begin{equation}\sum_{i=1}^n V_n \coloneq \set{v \in V \left\mid \exists v_1 \in V_1, \dots{}, v_n \in V_n : v = \sum_{i=1}^n v_i\right.}\,.\label{eqn:SumVectSpace}\end{equation}
Lo spazio~\eqref{eqn:SumVectSpace} è detto {\em diretto} quando per ogni \(v \in \sum_{i=1}^n V_i\) sono anche unici i vettori \(v_1 \in V_1, \dots{}, v_n \in V_n\) tali che \(v = \sum_{i=1}^n v_i\); in tal caso, questo spazio si scrive come \(\bigoplus_{i=1}^n V_i\). Questo spazio vettoriale ha sia funzioni componenti che iniezioni:
\begin{align*}
& p_j : \bigoplus_{i=1}^n V_i \to V_j\,, \ \sum_{i=1}^n v_i \to v_j \\
& q_j : V_j \to \bigoplus_{i=1}^n V_i\,, \ v \to v\,.
\end{align*}
Far vedere che \(\bigoplus_{i=1}^n V_i\) è un prodotto o un coprodotto in \(\Vect\) a seconda che lo si prenda in esame con \(\set{p_i}_{i=1, \dots, n}\) o \(\set{q_i}_{i=1, \dots, n}\). {\color{red} [Non è meglio che questo sia un esempio?]}
\end{esercizio}

\begin{esempio}
Sia \((X, \leqslant)\) un insieme preordinato, visto come categoria ovviamente; sia \(S \subseteq X\) è vediamo cosa sono, se esistono, prodotti e i coprodotti di \(S\).\newline
Anzitutto, un \(p \in X\) con una collezione di frecce \(p \to x\), una per ogni \(x \in S\) è un \(p \in X\) tale che \(p \leqslant x\) per ogni \(x \in S\). Quindi, un prodotto di \(S\) è un qualche \(p \in X\) soddisfacente le seguenti proprietà
%
\begin{enumerate}
\item \(p \leqslant x\) per ogni \(x \in S\)
\item per ogni \(c \in X\) per cui \(c \leqslant x\) per ogni \(x \in S\) si ha \(c \leqslant p\).
\end{enumerate}
%
Vediamo ora cosa può essere un coprodotto di \(S\). Un \(q \in X\) con frecce \(x \to q\), una per ogni \(x \in S\), è un \(q \in X\) tale che \(x \leqslant q\) per ogni \(x \in S\). Analogamente a prima, un coprodotto di \(S\) è un \(q \in X\) tale che
%
\begin{enumerate}
\item \(x \leqslant q\) per ogni \(x \in S\)
\item per ogni \(c \in X\) per cui \(x \leqslant c\) per ogni \(x \in S\) si ha \(q \leqslant c\).
\end{enumerate}
%
Nella Teoria degli Insiemi, i prodotti sono detti {\em estremi inferiori}, mentre i coprodotti sono chiamati {\em estremi superiori}. Come in generale, i prodotti e i coprodotti sono unici a meno di isomorfismi; tuttavia, nel caso in cui \((X, \leqslant)\) sia un insieme ordinato (cioè nel quale l'isomorfismo è uguaglianza), se esistono, i prodotti e i coprodotti sono unici. In questo caso, il prodotto e il coprodotto di \(S\) si indicarli con \(\inf S\) e \(\sup S\) rispettivamente.\newline
L'insieme \(\setR\) con l'usuale ordine \(\leq\) è un insieme ordinato. La cosa interessante di questo insieme è la completezza: ogni insieme inferiormente limitato ha estremo inferiore e ogni insieme superiormente limitato ha estremo superiore. Non è un fatto scontato questo: per esempio in \((\mathbb Q, \leq)\) queste cose non succedono!
\end{esempio}

\begin{esercizio}
Prendi \(\setN\) con la relazione di divisibilità \(\divides\):
\[\forall m, n \in \setN : m \divides n \xLeftrightarrow{\text{def}} \exists k \in \setN : mk = n.\]
Questa relazione è d'ordine. Dimostrare che esistono e sono unici i prodotti e i coprodotti di \((\setN, \divides)\). A proposito, cosa sono precisamente?
\end{esercizio}

%\begin{esempio}[Algebre di Heyting]
%Un {\em reticolo} è un insieme ordinato \((X, \leqslant)\) che ha prodotti e coprodotti di ciascuna coppia di suoi elementi. In questo contesto, il prodotto di due elementi \(a, b \in X\) si scrive come \(a \land b\), mentre il coprodotto si indica con \(a \lor b\). Un reticolo che ha anche oggetti terminale ed iniziale si dice {\em limitato}. È uso abbastanza diffuso indicare con \(\upvdash\) il minimo e \(\downvdash\) il massimo; esistono anche \(0\) e \(1\) per chiamare il minimo e il massimo.
%\end{esempio}

\begin{esercizio}
Questo esercizio è ambientato in una qualche categoria, non importa quale; il tutto accade lì dentro. Siano \(x_1\), \(x_2\) e \(x_3\) degli oggetti, \(x_1 \times x_2\) un oggetto con le frecce uscenti
\[x_1 \lfuncto{p_1} x_1 \times x_2 \functo{p_2} x_2\]
un prodotto di \(\set{x_1, x_2}\) e \((x_1 \times x_2) \times x_3\) un oggetto con le frecce
\[x_1 \times x_2 \lfuncto{p_{12}} (x_1 \times x_2) \times x_3 \functo{p_3} x_3\]
un prodotto di \(\set{x_1 \times x_2, x_3}\). Dimostrare che \((x_1 \times x_2) \times x_3\) con opportune frecce è un prodotto di \(\set{x_1, x_2, x_3}\). (Parte dell'esercizio è capire quali sono queste frecce.) Ora prendi un prodotto di \(\set{x_2, x_3}\) fatto di un oggetto chiamato \(x_2 \times x_3\) con frecce
\[x_2 \lfuncto{p_2} x_2 \times x_3 \functo{p_3} x_3\]
e un prodotto di \(\set{x_1, x_2 \times x_3}\) consistente di un oggetto scritto come \(x_1 \times (x_2 \times x_3)\) e di due frecce
\[x_1 \lfuncto{p_1} x_1 \times (x_2 \times x_3) \functo{p_{23}} x_2 \times x_3.\]
Come prima mostra che \(x_1 \times (x_2 \times x_3)\), affiancato da opportune frecce uscenti, è un prodotto di \(\set{x_1, x_2, x_3}\).\newline
Da questo esercizio deduci che
\[(x_1 \times x_2) \times x_3 \cong x_1 \times (x_2 \times x_3)\]
ovvero una sorta di \q{associatività del prodotto (binario)}.
\end{esercizio}

\begin{esercizio}
Riusciresti a fare una cosa simile per il coprodotto di tre oggetti?
\end{esercizio}
